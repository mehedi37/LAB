\documentclass{book}
\usepackage{pict2e, curves, graphicx, xcolor, subfigure, tabularx, wrapfig, lipsum}
\begin{document}
\begin{figure}[!htbp]
	\includegraphics[width=0.2\textwidth]{image.png} \hfill
	\includegraphics[width=0.2\textwidth]{image.png}
	\caption{A sample image}
	\label{fig:image}
\end{figure}
\begin{figure}[!htbp]
	\centering
	\includegraphics[angle=40, width=0.5\linewidth,height=0.5\linewidth]{image.png}
	\caption{A sample image at an angle of 40 degrees}
\end{figure}
% ------------------------Sub Figure--------------------------------
\newpage
\section*{\underline{Sub Figure:}}
\begin{figure}[!htbp]
	\centering
	\subfigure[First subfigure]
	{
		\includegraphics[width=0.2\textwidth]{image.png}
		\label{fig:first}
	} \hfill
	\subfigure[Second subfigure]
	{
		\includegraphics[width=0.3\textwidth]{image.png}
		\label{fig:second}
	} \\
	\subfigure[Third subfigure]
	{
		\includegraphics[width=0.5\textwidth]{image.png}
		\label{fig:third}
	}
	\caption{A sample image}
\end{figure}
% ------------------------Sub Figure--------------------------------

% ------------------------ReNew sub figure--------------------------
\newpage
\section*{\underline{ReNew sub figure:}}
\renewcommand{\thesubfigure}{(\roman{subfigure}).}
\begin{figure}[!htbp]
	\centering
	\subfigure[First subfigure]
	{
		\includegraphics[width=0.2\textwidth]{image.png}
		\label{fig:first}
	} \hfill
	\subfigure[Second subfigure]
	{
		\includegraphics[width=0.3\textwidth]{image.png}
		\label{fig:second}
	} \\
	\subfigure[Third subfigure]
	{
		\includegraphics[width=0.5\textwidth]{image.png}
		\label{fig:third}
	}
	\caption{A sample image}
\end{figure}
% ------------------------Wrap figure--------------------------
\newpage
\section*{\underline{Wrap figure:}}
% \begin{wrapfigure}[10]{r}{0.5\textwidth}		% 10: number of lines to be wrapped
\begin{wrapfigure}{r}{0.5\textwidth}
	\centering
	\includegraphics[width=0.5\textwidth]{image.png}
	\caption{A sample image}
\end{wrapfigure}
\lipsum[1]
% ------------------------Wrap figure --------------------------

% ------------------------Picture Draw--------------------------
\chapter{Draw Figures}
\section{Picture}
\begin{picture}(100,100)(0,0)		% (100,100): size of the picture, (0,0): position of the picture
	\put(0,0){\line(1,0){100}}
	\put(0,0){\line(0,1){100}}
	\put(100,0){\line(0,1){100}}
	\put(0,100){\line(1,0){100}}
	\put(50,50){\circle{50}}

	\put(25,60){\circle{25}}
	\linethickness{4pt}
	\put(25,60){\textcolor{red}{\circle{15}}}

	\linethickness{1pt}
	\put(75,60){\circle{25}}
	\linethickness{4pt}
	\put(75,60){\textcolor{red}{\circle{15}}}

	\linethickness{1pt}
	\put(50,50){\circle{25}}
	\put(50,50){\circle*{20}}
\end{picture}
% ------------------------Picture Draw--------------------------

% ------------------------ Line --------------------------
\section{Line}
\setlength{\unitlength}{0.75mm}
\begin{picture}(60,30)(0,0)		% (60,30): size of the picture, (0,0): position of the picture
\put(5,5){\line(0,1){20}}
\put(15,15){\line(1,0){20}}
\put(45,25){\line(1,-1.5){15}}	% (1,-1.5) = slope & 15 = length of the line
\thicklines
\put(15,-1){\vector(1,0){40}}
\end{picture}
% ------------------------ Line --------------------------
\section{MultiPuts}
\begin{picture}(60,30)(0,0)		% (60,30): size of the picture, (0,0): position of the picture
\multiput(0,0)(10,0){3}{\circle*{2}}
\multiput(0,10)(10,0){3}{\circle*{2}}
\multiput(0,20)(10,0){3}{\circle*{2}}
\multiput(10, -5)(10,10){3}{\line(1, -1){8}}
\multiput(10, -5)(10,10){3}{\vector(-1, -1){8}}
\end{picture}
% ------------------------ Line --------------------------

% ------------------------ Make box --------------------------
\section{Make box}
\begin{picture}(60,30)(0,0)		% (60,30): size of the picture, (0,0): position of the picture
\put(0,0){\framebox(60,30){}}
\put(0,0){\framebox(30,15){}}
\put(30,15){\framebox(30,15){}}
\end{picture}
\\[2cm]
\setlength{\unitlength}{1mm}
\begin{picture}(55,60)(0,0)
\put(5,55){\framebox(20,5)[c]{(5,55)}}
\put(5,45){\framebox(20,5)[t]{(5,45)}}
\put(5,35){\framebox(20,5)[b]{(5,35)}}
\put(5,25){\framebox(20,5)[l]{(5,25)}}
\put(5,15){\framebox(20,5)[r]{(5,15)}}
\put(5,5){\framebox(11,5)[l]{Square box}}
%
\put(30,55){\dashbox{1.5}(20,5)[c]{(30,55)}}
% 1.5: length of the dashes, (20,5): size of the box,
% [c]: position of the text, (30,55): it is the display text
\put(30,45){\dashbox{0.5}(20,6)[t]{(30,45)}}
\put(30,35){\dashbox{0.5}(20,5)[b]{(30,35)}}
\put(30,25){\dashbox{0.5}(20,5)[l]{(30,25)}}
\put(30,15){\dashbox{0.5}(20,5)[r]{(30,15)}}
\put(30,5){\dashbox{0.5}(11,5)[l]{Square box}}
\end{picture}
% ------------------------ Make box --------------------------

% ------------------------ Complex Fig --------------------------
\section{Complex Fig}
\begin{picture}(20,20)(0,0)
	\multiput(5,5)(5,10){2}{\line(1,0){20}}
	% (5,5): start point, (5,10): slope, {2}: number of lines, {20}: length of the line
	% (5, 5) = starting point
	% (5, 10) = distance between two lines
	\multiput(5,5)(20,0){2}{\line(1,2){5}}
\end{picture}

\begin{picture}(85,45)(0,0)
	\curve(0,20, 5,40, 10,20)
	\curve(10,20, 15,0, 20,20)
	\curve(20,20, 25,35, 30,20)
	\curve(30,20, 35,5, 40,20)
	\curve(40,20, 45,30, 50,20)
	\curve(50,20, 55,10, 60,20)
	\curve(60,20, 65,25, 70,20)
	\curve(70,20, 75,15, 80,15)
\end{picture}
\end{document}