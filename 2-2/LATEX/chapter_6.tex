\documentclass[a4paper, 10pt]{book}
\usepackage{ enumerate, tabularx, asymptote, amsmath, amssymb, amsfonts, geometry, color, setspace}
\usepackage{pdflscape, rotating}

\begin{document}
\tableofcontents
\chapter[Listing Items]{Listing}
\section[Enumerate]{Enumerate}
\begin{table}[!h]
	\begin{tabularx}{10cm}{|X|X|}
		\hline
		code & output \\
		\hline
		\textbackslash begin\{enumerate\}   \newline
		\textbackslash item First   \newline        % using \\ will change the input column
		\textbackslash item Second  \newline
		\textbackslash item Third   \newline
		\textbackslash end\{enumerate\} \newline
		     &
		List will look like this
		\begin{enumerate}
			\item First
			\item Second
			\item Third
		\end{enumerate}
		\\
		\hline
	\end{tabularx}
\end{table}

\section[Nested Enumerate]{Nested Enumerate}
\begin{table}[!h]
	\begin{tabularx}{15cm}{|X|X|}
		\hline
		\textbf{code} & \textbf{output} \\
		\hline
		\textbackslash begin\{enumerate\}   \newline
		\textbackslash item First   \newline        % using \\ will change the input column
		\textbackslash item Second  \newline
		\textbackslash item Third   \newline
		\textbackslash begin\{enumerate\}   \newline
		\textbackslash item First   \newline        % using \\ will change the input column
		\textbackslash begin\{enumerate\}   \newline
		\textbackslash item Sub First   \newline        % using \\ will change the input column
		\textbackslash begin\{enumerate\}   \newline
		\textbackslash item Sub Sub First   \newline        % using \\ will change the input column
		\textbackslash begin\{enumerate\}   \newline
		\textbackslash item Sub Sub First   \newline        % using \\ will change the input column
		\textbackslash item Sub Sub Second   \newline        % using \\ will change the input column
		\textbackslash end\{enumerate\} \newline
		\textbackslash item Sub Second   \newline        % using \\ will change the input column
		\textbackslash item Sub Third   \newline        % using \\ will change the input column
		\textbackslash end\{enumerate\} \newline
		\textbackslash item Second  \newline
		\textbackslash item Third   \newline
		\textbackslash end\{enumerate\} \newline
		\textbackslash end\{enumerate\} \newline
		     &
		List will look like this
		\begin{enumerate}
			\item First
			\item Second
			\item Third
			      \begin{enumerate}
				      \item Sub First
				            \begin{enumerate}
					            \item Sub First Sub
					                  \begin{enumerate}
						                  \item Sub Sub First
						                        % \begin{enumerate} % This will give error
						                        %     \item Super Sub First
						                        %     \item
						                        % \end{enumerate}
						                  \item Sub Sub Second
					                  \end{enumerate}
					            \item Sub Second
					            \item Sub Third
				            \end{enumerate}
				      \item Second
				      \item Third
			      \end{enumerate}
		\end{enumerate}
		\\
		\hline
	\end{tabularx}
	\caption{Nested Enumerate}
\end{table}

\chapter[Unordered List]{Unordered List}
\section{Itemize}
\begin{table}[!h]
    \begin{tabularx}{15cm}{|X|X|}
        \hline
        \textbf{code} & \textbf{output} \\
        \hline
        \textbackslash begin\{itemize\}   \newline
        \textbackslash item First   \newline        % using \\ will change the input column
        \textbackslash item Second  \newline
        \textbackslash item Third   \newline
        \textbackslash end\{itemize\} \newline
             &
        List will look like this
        \begin{itemize}
            \item First
            \item Second
            \item Third
        \end{itemize}
        \\
        \hline
    \end{tabularx}
		\caption Itemize
\end{table}

\section{Nested Itemize}
\begin{itemize}
	\item First
	\begin{itemize}
		\item Sub First
		\item Sub Second
		\begin{itemize}
			\item Sub Sub First
			\item Sub Sub Second
			\begin{itemize}
				\item Sub Sub Sub First
				% \begin{itemize}
				% 	\item Sub Sub Sub First
				% \end{itemize}
				\item Sub Sub Sub Second
			\end{itemize}
			\item Sub Sub Third
			\item Sub Sub Fourth
		\end{itemize}
		\item Sub Third
	\end{itemize}
	\item Second
	\item Third
\end{itemize}

\subsection{Point Names in \textcolor{red}{Itemize}}
\begin{itemize}
	\item[$\bullet$] bullet
	\item[$\circ$] circ
	\item[$\ast$] ast
\end{itemize}

\subsection{Unordered Points in \textcolor{red}{Description}}
\begin{description}
	\item[$\bullet$] bullet
	\item[$\rightarrow$] rightarrow
	\item[$\leftarrow$] leftarrow
	\item[$\Rightarrow$] Rightarrow
	\item[$\Leftarrow$] Leftarrow
	\item[$\diamond$] diamond
	\item[$\Leftrightarrow$] Leftrightarrow
	\item[$\leftrightarrow$] leftrightarrow
	\item[$\blacksquare$] blacksquare
	\item[$\bigstar$] bigstar
	\item[$\Rightarrow$] Rightarrow
	\item[--] hyphen (--)
	\item[NOTE:] This is not a complete list
	\item[!]  Exclamation mark
\end{description}

\subsection{Ordered Points in \textcolor{red}{Description}}
\begin{description}
	\item[1.] First
	\item[2] Second
	\item[(3)] Third
	\item[(a)] a with round bracket
	\item[(B)] B with round bracket
	\item[(i)] roman with round bracket
	\item test, will it continue from previous number
	\item[!] \textcolor{red}{No it will not continue from previous number.}
\end{description}

\chapter[ Renewcommand In List]{Renewcommand In List}
\section{Renewcommand Type: 1, In \textcolor{red}{Enumerate}}

\subsection{Understandin Renewcommand In \textcolor{red}{Enumerate}}
\textbf{Note:} \textcolor{red}{\textbackslash renewcommand}
\{\textbackslash labelenumi\}\{\textbackslash arabic\{enumi\}.\}
\begin{itemize}
	\item \textbackslash label \textcolor{red}{\textbf{enum}} = enum and \textcolor{red}{\textbf{i}} = level of list i = 1, ii = 2, iii = 3, iv = 4
	\item \textbackslash arabic\{enumi\} = 1, 2, 3, 4
	\item \textbackslash alph\{enumi\} = a, b, c, d
	\item \textbackslash Alph\{enumi\} = A, B, C, D
	\item \textbackslash roman\{enumi\} = i, ii, iii, iv
	\item \textbackslash Roman\{enumi\} = I, II, III, IV
	\item Now the separators are used as dot (.) or bracket ().
	\item like \textbackslash labelenumi\{\textcolor{red}{(}\textbackslash alph\{enumi\}\textcolor{red}{)}\} will give (a), (b), (c), (d) \\\\
\end{itemize}


\subsection{Renewcommand Example In \textcolor{red}{Enumerate}}
\renewcommand{\labelenumi}{\Roman{enumi}.}
\renewcommand{\labelenumii}{\Roman{enumi}.(\arabic{enumii})}
\renewcommand{\labelenumiii}{\Roman{enumi}.(\arabic{enumii})\arabic{enumiii}}
\renewcommand{\labelenumiv}{\Roman{enumi}.(\arabic{enumii})\arabic{enumiii}.(\arabic{enumiv})}

\begin{enumerate}
	\item The Dot at the last of renewcommand is required to declare how the list will be presented
	\begin{enumerate}
		\item Sub First
		\item Sub Second
		\begin{enumerate}
			\item Sub Sub First
			\item Sub Sub Second
			\begin{enumerate}
				\item Sub Sub Sub First
				\item Sub Sub Sub Second
			\end{enumerate}
			\item Sub Sub Third
			\item Sub Sub Fourth
		\end{enumerate}
		\item Sub Third
	\end{enumerate}
	\item Second
	\item Third
\end{enumerate}

\section{Renewcommand type: 2 In \textcolor{red}{Enumerate}}
\subsection{  \textcolor{red}{\textbackslash theenum}  in Renewcommand}
\textbf{Command:} \textbackslash renewcommand\{\textbackslash labelenumi\}\{Q.\textbackslash theenumi: \}
\begin{enumerate}
	\item \textbackslash theenum = number of list
	\item Q. is the prefix of the list = Q.1, Q.2, Q.3, Q.4
\end{enumerate}
\subsection{Renewcommand Example In \textcolor{red}{Enumerate}}

\renewcommand{\labelenumi}{Q.\theenumi:}
\renewcommand{\labelenumii}{.\theenumii:}

\textbf{Question of CT:}
\begin{enumerate}
	\item First Question 	\hfill 5 marks
	\begin{enumerate}
		\item This will be the same as the previous, because we used for enum\textcolor{red}{i} only
	\end{enumerate}
	\item Second Question	\hfill 5 marks
	\item Third Question	\hfill 5 marks
\end{enumerate}

\newpage
\section{New Numbering style: }
\begin{enumerate}[{\bf Q.1 :}]
	\item test
	\item test2
	\begin{enumerate} [{\bf {A}ns (a):}]
		\item Ans1
		\begin{enumerate}	[{\bf Sub {A}ns (i):}]
			\item sub ans 1
			\item sub ans 2
		\end{enumerate}
		\item Ans2
	\end{enumerate}
\end{enumerate}

\newpage
\section{Renewcommand In \textcolor{red}{Itemize}}

\textbf{Code:} \textbackslash renewcommand\{\textbackslash labelitemi\}\{\textbackslash textbf\{--\}\}
\begin{itemize}
	\item label and then \textcolor{red}{item} and then \textcolor{red}{i} = 1st level of items
\end{itemize}

\renewcommand{\labelitemi}{\textcolor{red}{\textbf{--} $\rightarrow$}}	% multiple types can be combined
\renewcommand{\labelitemii}{\textcolor{red}{$\bigstar$}}	% previous levels types can also be used
\renewcommand{\labelitemiii}{\labelitemi \textcolor{yellow}{$\Diamond$}}  % try $\diamond$
\renewcommand{\labelitemiv}{\textcolor{yellow}{$\blacksquare$}}


\begin{itemize}
	\item Multiple types can be combined
	\item It has \textbf{Double Hyphen (-\@-)} and \verb|$\rightarrow\$| commands combined
	\begin{itemize}
		\item Sub First
		\item Sub Second
		\begin{itemize}
			\item Previous levels types can also be used
			\item Sub Sub Second
			\begin{itemize}
				\item \verb|$\blacksquare$| with color yellow
				\item Sub Sub Sub Second
			\end{itemize}
			\item Sub Sub Third
			\item Sub Sub Fourth
		\end{itemize}
		\item Sub Third
	\end{itemize}
	\item Third
\end{itemize}



% Tabbing Section
\chapter[Tabbing]{Tabbing}
\section{Tabbing}
\begin{table}[!h]
    \begin{tabularx}{15cm}{|X|X|}
        \hline
        \textbf{code} & \textbf{output} \\
        \hline
        \textbackslash begin\{tabbing\}   \newline
        \textbackslash hspace\{2cm\} \textbackslash = \textbackslash hspace\{2cm\} \textbackslash= \textbackslash hspace\{2cm\} \textbackslash= \textbackslash kill \newline
        1\textbackslash =~First \textbackslash kill \verb|\= is only for the 1st item|  \newline   % using \\ will change the input column
        2\textbackslash \textgreater Second \textbackslash kill \newline  % using \\ will change the input column
        3\textbackslash \textgreater Third \textbackslash kill \newline   % using \\ will change the input column
        \textbackslash end\{tabbing\} \newline

             &
        List will look like this
        \begin{tabbing}
            \hspace{2cm} \= \hspace{2cm} \= \hspace{2cm} \= \kill
            1\> First \\
            2\> Second \\
            3\> Third   \\
            \>1\'2 \\
        \end{tabbing}
        \\
        \hline
    \end{tabularx}
\end{table}
\end{document}