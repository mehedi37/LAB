\documentclass[a4paper, 10pt]{article}
\usepackage{tabularx, graphicx, amsmath, amssymb, amsfonts, geometry, color}
\begin{document}
Hello World!
\LaTeX \ is fun.
\textcolor{red}{\copyright MeEk\_0}
\\
MeEk\_0     % For table
\\\\      % For new line
$x^2 \ =\ 4$ \\
{\color{red} $x^2 \ =\ 4$} \\
{\color{blue} $x^2 \ =\ 4$} \\

\begin{align}
	\mathit{x^2 = 4} \\
	\mathit{x^2 = 4}
\end{align}

\begin{equation}
	x^2 = 4	\\
	x^3 = 8
\end{equation}
\\
% 2.1 chapter
\textbf{A bold face Text} \\
\textit{An italicized Text} \\
\textsl{A slanted Text} \\
\textsc{A small caps Text} \\
\uppercase{A uppercase Text} \\
\lowercase{A LOwERCASe TexT} \\
\emph{A emphasized Text} \\
\\
\texttt{A typewriter Text} \\
\textrm{A Serif family Text} \\
\textsf{A Sans Serif family Text} \\
\\
% Math fonts
$\mathrm{Sans}$ \\
$\mathit{Italic}$ \\
$\mathbf{Bold}$ \\
$\mathsf{Sans Serif}$ \\
$\mathtt{Typewriter}$ \\
$\mathcal{CALLIGRAPHIC}$ \\
$\mathfrak{Fraktur \ German}$ \\
\include{chapter1}		% Include chapter1.tex

\underline{Underline} \\
\verb|text| \\
\verb "$$"\\
$X_2 = 2$ \\  		% subscript
$X^2 = 2$ \\   		% superscript
$\backslash \ backslash $   \\
$\sim \ Binds two words to be printed in the same line. $ 	\\

\begin{table}[h]		% h = here, t = top, b = bottom, p = page
	\begin{tabularx}{5cm}{|X|X|X|}
		\hline      % to create a horizontal line
		H1 & H2 & H3 \\
		\hline
		1  & 2  & 3  \\
		4  & 5  & 6  \\
		7  & 8  & 9  \\
		\hline
	\end{tabularx}
\end{table}
\end{document}